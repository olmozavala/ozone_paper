\section{Conclusions}

The described work is a new operational system for the prediction
of ozone 24 hours ahead of the current time. The system is based
on shallow neural networks and uses meteorological data from a forecast system
and air quality data from the RAMA to train the network. It was tested
training the model for the years 2000 to 2016 and evaluated on the year 2017. 


\ldots
In the near future we will investigate the use of deeper architectures to improve
our predictions. In our current approach only 96 values are been using for each
meteorological variable, using Convolutional Neural Networks (CNNs) with residual 
connections \cite{deep} may allow the system to learn from a much larger dataset. 

% Summary of how well thus the system works, and say that this type of systems are a good
% alternative to deterministic models (Pablo, Olmo, Jorge)


