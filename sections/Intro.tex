\section{Introduction}
Air pollution is a major issue for public health that can induce cancer and mortality 
\cite{Pope2002,SMITH20092091}. In Mexico City, ozone concentrations systematically exceed the 
health official standard (58\% of the days in 2016 and 2015) \cite{sedema_dias2016}. The 
risk can be reduced by lower ambient concentrations or by reducing exposure to pollutants. 
In the case of Mexico City an alert is issued after measuring high ozone 
levels during an hour, in which case it is not possible to protect the population for 
the already exposure time period. In order to reduce the hazards of air pollutants an
operational forecast system is an essential tool.

% A summary of the sate of the art in air quality forecast methods (Agustin)
\subsection{Air quality forecast systems}

In Mexico  the only official air quality forecast is issued by the Mexico City Environmental 
Agency (SEDEMA) since February 2017. It uses the WRF-ARW model (Weather Research and 
Forecasting Model-Advanced Research) \cite{huang2008description} for meteorology, HERMES-Mex 
(High-Elective Resolution Modeling Emission System for Mexico) as emissions processing tool 
\cite{GUEVARA2017882} and CMAQ (Community Multi-scale Air Quality) \cite{byun2006review} for 
its chemical and transport model. Up to now there is not a forecast performance official report for 
this system.
Other air quality forecast system uses the MCCM model, MCCM is a three dimensional meteorology 
model based on the Penn State/ NCAR meteorology model (MM5) with coupled air chemistry  
\cite{GRELL20001435}, the emissions are for 2005 year, uses the RADM2 chemical mechanism 
and has been issued by the Atmospheric Science Center at UNAM since 2006 \cite{hernandez2006} 
and it has been evaluated in 2009 \cite{resendiz2009}. The evaluation performance shows 
an enhancement when including emissions from surrounding cities such as Toluca and Cuernavaca 
\cite{garcia2009ozone}. For ozone, the index of agreement has an average of 0.81 in 
this forecast system. An update of this system uses the WRF-chem \cite{grell2005fully} 
and the 2008 emissions inventory, up to now there is not an evaluation performance 
of this new system.

The use of Neural Networks (NN) and other machine learning algorithms for the prediction of 
air quality have some good results.  One of the first approaches was developed
for Athens in 2006, NNs were used for the prediction of hourly concentrations of
$PM_{10}$ \cite{grivas_artificial_2006}.  the development made 
in Chile \cite{perez_integrated_2006} 
used NN for the same pollutants. 
In Mexico, there is a system for the prediction of Ozone in the 
city of Guadalajara \cite{ignacio_garcia_pronostico_nodate}.
Other developments focus on the prediction in specific areas, such as the prediction of 
pollutants near urban arterial of China \cite{cai_prediction_2009}, other approaches in the 
prediction of the quality index, seek to find which are the best parameters to train the 
network, using other techniques like SOM \cite{kolehmainen_neural_2001} or genetic algorithms 
\cite{niska_evolving_2004}.
Some more advanced development are held in China \cite{jiang_progress_2004} and the city of 
Perugia,Italy \cite{viotti_atmospheric_2002}

% Introduction to air quality systems using neural networks (Agustin, Pablo)

