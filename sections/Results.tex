\section{Operational Mode}
The system was tested for the year 2017 and it is now been used as an operational forecast
system since January $1^{\text{st}}$ 2018. The operational system makes predictions every hour and
the weights of the NNs are updated once every month. The original approach was to update the weights
of the NNs every hour, but we realized that current data that comes from the RAMA stations 
have several errors and missing data. To access information from RAMA there are two main choices, download the monthly data of
previous months or download the latest hourly data of the current month. However, the data
from previous months has a version with a quality control process and do not contains the 
same values. This posed 
a major problem to our system because the NNs are trained with historic (nicely cleaned) data, and
the operational system uses the current (raw) data.  

To improve the performance of the operational system an hourly climatology for each month 
of the year was computed for each pollutant, using the historic data from 2000 to 2017. When 
there is missing values in the raw data of the RAMA stations, the system uses the values of 
the climatology to make its predictions. 

The operational forecast system is available at \url{http://pronosticos.unam.mx/contaminantes/}.

\section{Results}
\label{sec:results}

% Explain the different options that we tried so that we can see the improvements (Pablo,Olmo)
% Which metrics are we using

All calculations of this sytem were performed on a multy GPU system with four Nvidia GeForce GTX 1080 
graphics cards at 1.9 GHz and 8GB of RAM, with an Intel Xeon E52630 as the main processors, and 64 GB of RAM. 

% MENTION CUDA LIBRARY VERSION AND SPECS OF TENSOR FLOW, WRF VERSION

As mentioned before, in order to quantify the performance of the system it was first tested
 by training the NNs with data from the years 2000 to 2016 and evaluated for
the year 2017. In the final operational system, the NNs are trained with data up to the year 2017
and used to make predictions for the year 2018. 

Figure X shows the average correlation coefficent obtained for each station during the 2017 year. The 
mean correlation coefficient is Y and the mean 


Figure X shows the first week of June for 

In Mexico City the quality of the air is categorized in four types: good, moderate, bad, very bad, and
extremely bad. These categories are related to the amount of particles in the air of ozone, $PM_{10}$, or
$PM_{2.5}$. In the specific case of ozone, the thresholds are: 0-70 (good), 71-95 (moderate), 
(96-154) bas

Table X shows the overall contingency table. This table allows us to measure the performance
of the system to correctly predict when the air quality will be above the safe levels. In this case
if any station surpasses the 155 ppm then it is a treath. A true positive is when any of the 
RAMA stations predicted a value of ozone above 155 ppm and the proposed system also
predicted a value of ozone above 155 ppm for any station. 




