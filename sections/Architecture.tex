\subsection{Neural Network Architecture}

The proposed system includes a set of 34 Neural Networks (NNs) that are used 
to predict the Ozone pollutant 24 hrs after the present time. 
Each Neural Network is responsible for forecasting Ozone for one of the RAMA's stations. 
One training example in our system has air quality information 
at time $t$ and forecasted metorological variables at times $t, t+1, t+2,\cdots t+24$. 
The output of the model is the ozone value at time $t+24$, which allows the
goverment to decide if an air quality contingency is necessary one day in advance.
This knowledge could avoid the population getting affected by high indices of pollution.

The forecast model is implemented using Python and the machine learning library Tensor Flow 
\cite{tensorflow2015-whitepaper}. The proposed layout for each NN
is a shallow architecture with two hidden layers and one output layer, all the layers have dense 
connections and a sigmoid function as the activation function. 
 The first hidden layer has $N$ units (neurons), where $N$ is the number of input features. The 
second hidden layer has $2*N$ units, and the output layer has one output unit which predicts the 
ozone value in 24 hrs. Each layer has equal number of bias units as input units. The
bias units are used to reduce overfitting.
The weights of all units, normal units and bias units, are initialized randomly.
In summary the activation function of a single unit in the first hidden layer becomes:
\begin{equation}
	f_j(W,X) = \sigma (\Sigma_{i=1}^N w_{j, i}x_i + b_i )
\end{equation}
Where $\sigma$ is the sigmoid functoin, $b_i$ are the weights of the bias units and
$W$ and $X$ are the weights and activation values of the input layer.  Finally, the 
optimization algorithm that we use is the Adam Optimizer \cite{adam}.  

The proposed architecture is straightforward, the novelty of our approach
lies on using meteorological data from a forecast model merged with current
air quality data to make a prediction for ozone. 
